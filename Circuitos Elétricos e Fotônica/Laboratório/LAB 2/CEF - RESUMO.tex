\documentclass[a4paper,12pt,onecolumn]{article}
\usepackage[utf8]{inputenc}
\setlength{\columnsep}{1cm}
\usepackage[utf8]{inputenc}
\usepackage[brazil]{babel}
\usepackage{amsmath}
\usepackage{bbm}
\numberwithin{equation}{section}
\numberwithin{figure}{section}
\usepackage{hyperref}
\usepackage{graphicx}
\usepackage[stable]{footmisc}
\usepackage{xcolor,colortbl}
\addtolength{\hoffset}{-0.5cm}
\addtolength{\textwidth}{1.5cm}
\usepackage{bigints}
\usepackage{comment}
\usepackage{abstract}
\usepackage{indentfirst}
\renewcommand{\abstractnamefont}{\normalfont\Large\bfseries}
\renewcommand{\abstracttextfont}{\normalfont}


\begin{document}
\begin{figure}[hbtp]
\centering
\includegraphics[scale=0.5]{ufabc.jpg}
\end{figure}

\begin{center}
\textbf{Universidade Federal do ABC}
\end{center}

\vspace{0.5cm}

\hrule
\begin{center}
\textbf{Circuitos Elétricos e Fotônica - Resumo}
\end{center}
\hrule


\vspace{8cm}

\begin{center}
São Bernardo do Campo - 2020
\end{center}

\newpage

\begin{abstract}

\end{abstract}


\newpage
\begin{center}
\tableofcontents
\end{center}

\newpage
\section{Carga Elétrica}
A carga elétrica pode ser classificada da

Bipolar
Quantizada

\begin{equation}
1 \text{ C} = 6.242 \times 10^{18} \text{ elétrons}
\end{equation}{}

Com relacao ao Princípio da conservação de carga Lei de Coulomb

\begin{equation}
\mid \vec{F} \mid = \dfrac{\mid Q_{1} \mid \times \mid Q_{2} \mid}{r^2}
\end{equation}{}

Materiais
Codutores: As cargas elétricas se deslocam de maneira relativamente livre
Isolantes:
Semicondutores:

\begin{center}
  \href{www.google.com}{Lista 1}
\end{center}






\end{document}
